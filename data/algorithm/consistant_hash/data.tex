\begin {document}

\title {\ZHH \huge 一致性哈希 Consistant Hash}
\author {\small gaccob}
\date {\small 2013 年 06 月 26 日}
\maketitle

{\ZHH 在动态变化的cache环境中,hash算法应该满足4个适应算法: }
\begin {itemize}
    \item { \textcolor{blue}{balance}, 尽量做到负载均衡. }
    \item { \textcolor{blue}{monotonicity}, 当cache节点变化时, 保护已分配的内容不重新mapping, 当然, 只能是尽量, 因为没有办法完全做到, 不然这个cache节点就没有存在的意义了. }
    \item { \textcolor{blue}{spread}, 当cache节点不是完全可见时, 会导致相同的内容mapping到不同的节点中去, 这就是spread, 要尽量避免这种情况. }
    \item { \textcolor{blue}{load}, 是指每个cache节点的负载要尽量低. }
\end {itemize}

\vspace {10pt}
{ \ZHH 大规模的分布式cache系统中, key-value如何做hash? }
\begin {itemize}
    \item { 最常规的莫过于hash计算得到一个整数再取模. }
    \item { 在cache节点变化的时刻, 会导致大量的cache不命中, 需要重新建立mapping关系, 这显然不是个好主意. }
\end {itemize}

\vspace {10pt}
{ \ZHH 一致性哈希: Consistant Hash }
\begin {enumerate}
    \item { Consistant Hash带来了什么? 在移除, 添加一个cache时, 它能够尽可能小的改变已存在key映射关系, 尽可能的满足monotonicity的要求. }
    \item { Consistant Hash的原理: 选择具体的cache节点不再只依赖于key的hash计算, cache节点本身也参与了hash计算. 参考文章: \href{http://www.akamai.com/dl/technical_publications/ConsistenHashingandRandomTreesDistributedCachingprotocolsforrelievingHotSpotsontheworldwideweb.pdf}{Consistant Hash and Random Trees}. }
    \item { 计算步骤: }
        \begin {itemize}
        \item { 将整个哈希值空间组织成一个虚拟的圆环, 如假设某哈希函数H的值空间为$[0, 2^{32}-1]$, 即哈希值是一个32位无符号整形, 整个空间按顺时针方向组织, 0和$[2^{32}-1]$在零点中方向重合. }
        \item { 对各个cache节点进行hash, key可以参考ip + 节点名, 并落在圆环上. }
        \item { 对于数据key做相同的hash计算, 并确定在圆环上的位置, 从此位置顺时针“行走”, 遇到的第一个cache节点就是mapping到的节点. }
        \end {itemize}
    \item { 容错性与扩展性分析: }
        \begin {itemize}
        \item { 某个cache节点宕机时, 只有该节点到它之面的第一个节点中间的数据受影响, 需要做remapping. }
        \item { 当增加cache节点时, 也只有该节点和它之前的第一个节点中间的数据受影响, 需要做remapping. }
        \end {itemize}
    \item { 简单的C代码参考(因为是链表组织的, 所以效率一般, 只做简单参考): }
\begin{lstlisting}[language={[ANSI]C}]
typedef struct conhash_t
{
    struct list_head node_list;
    hash_func key_hash;
    hash_func node_hash;
} conhash_t;

#define CONHASH_NODE_NAME_SIZE 128

struct conhash_node_t
{
    struct list_head link;
    void* data;
    uint32_t hash_value;
};

void* conhash_node(conhash_t* ch, void* key)
{
    uint32_t val;
    struct conhash_node_t* n;
    struct list_head* l;
    if (!ch || !key) return NULL;
    if (list_empty(&ch->node_list)) return NULL;

    val = ch->key_hash(key);
    list_for_each_entry(n, struct conhash_node_t, &ch->node_list, link) {
        if (n->hash_value >= val) return n->data;
    }
    l = ch->node_list.next;
    n = list_entry(l, struct conhash_node_t, link);
    return n->data;
}
\end{lstlisting}
\end {enumerate}

\end {document}
