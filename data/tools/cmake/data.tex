\begin {document}
\title {\ZHH \huge CMake使用}
\author {\small gaccob}
\date {\small 2013 年 9 月 12 日}
\maketitle

\section{\ZHH 什么是CMake}{
{\href{http://www.cmake.org/cmake/help/v2.8.11/cmake.html}{CMake}是一个跨平台的自动化建构系统, 它使用一个名为CMakeLists.txt的文件来描述构建过程, 可以产生标准的构建文件, 如Unix的Makefile或Windows Visual C++的projects/workspaces. 它的强大之处在于: 跨平台, 自动化. }\par
}

\section{\ZHH 常用的语法规则}{
\begin {table} [htbp]
\centering
\begin {tabular} {| c | c |}
    \hline
    注释        &   \#   \\
    \hline
    设置变量    &   set(var, value) \\
    \hline
    条件判断    &   if() else() endif() \\
    \hline
    for循环     &   foreach(loopvar, arg1, arg2, ...) endforeach() \\
    \hline
    while循环   &   while(condition) endwhile(condition)\\
    \hline
\end {tabular}
\end {table}
}

\section{\ZHH 常用的命令}{
\begin{lstlisting}[language=bash]
# `指定cmake的版本依赖`
cmake_minimum_required(**)

# `指定项目名称`
project(**)

# `指定头文件的搜索路径, 相当于指定gcc的-I参数`
include_directories

# `动态链接库或静态链接库的搜索路径, 相当于gcc的-L参数`
link_directories

# `包含子目录`
add_subdirectory

# `添加编译参数, 例如add\_definition(“-Wall -ggdb -O0″)`
add_definitions

# `添加链接库, 貌似可以不区分是共享库或者静态库, cmake会自动查找`
target_link_libraries

# `编译可执行文件`
add_executable

# `编译lib`
add_library

# `打印日志到终端`
message([SEND_ERROR | STATUS | FATAL_ERROR], `“***”`)
\end{lstlisting}
}

\section{\ZHH 常用的内部变量}{
\begin{lstlisting}[language=bash]
# `C编译器`
MAKE_C_COMPILER

# `C编译选项`
CMAKE_C_FLAGS

# `C++编译器`
CMAKE_CXX_COMPILER

# `C++编译选`
CMAKE_CXX_FLAGS

# `可执行文件的存放路径`
EXECUTABLE_OUTPUT_PATH

# `库文件路径`
LIBRARY_OUTPUT_PATH

# `build 类型, 可以指定Debug或者Release`
CMAKE_BUILD_TYPE

# `库类型, 可以指定动态库或者静态库(ON/OFF)`
BUILD_SHARED_LIBS

# `项目的根目录, 可以在在CMakeLists.txt中set设置, 也可以cmake时-D指定`
CMAKE_SOURCE_DIR
\end{lstlisting}
}

\section{\ZHH 其他的一些用法}{
\begin {itemize}
\item { 根据OS指定编译选项: if(APPLE); IF(UNIX); if(WIN32) }\par
\item { 字符串比较: if(** STREQUAL **) … endif() }\par
\end {itemize}
}

\section{\ZHH \href{https://github.com/gaccob/gbase}{gbase}中的sample}{
{CMakeLists.txt:} \par
\begin{lstlisting} [language=bash]
cmake_minimum_required(VERSION 2.8.10)
project(gbase)
include("${CMAKE_SOURCE_DIR}/gbase.cmake")
\end{lstlisting}

{common.cmake:} \par
\begin{lstlisting} [language=bash]
# `禁止共享库`
option(BUILD_SHARED_LIBS "build shared libraries." OFF)

# `编译类型`
if (CMAKE_CONFIGURATION_TYPES)
    message(STATUS "build type: ${CMAKE_CONFIGURATION_TYPES}")
elseif (CMAKE_BUILD_TYPE)
    message(STATUS "build type: ${CMAKE_BUILD_TYPE}")
else()
    set(CMAKE_BUILD_TYPE "Debug")
    message(STATUS "build type: ${CMAKE_BUILD_TYPE}")
endif()

# `编译选项(仅gcc、clang、vc)`
if ("${CMAKE_CXX_COMPILER_ID}" STREQUAL "GNU")
    set(COMMON_DEBUG "-ggdb -Wall -Werror -pg -O0")
    set(COMMON_RELEASE "-ggdb -Wall -Werror -pg -O1")
    set (COMMON_LINK_LIB z dl pthread m)
elseif ("${CMAKE_CXX_COMPILER_ID}" STREQUAL "Clang")
    set(COMMON_DEBUG "-ggdb -Wall -Werror -pg -O0")
    set(COMMON_RELEASE "-ggdb -Wall -Werror -pg -O1")
    set (COMMON_LINK_LIB z dl pthread m)
elseif ("${CMAKE_CXX_COMPILER_ID}" STREQUAL "MSVC")
    set(COMMON_DEBUG "/Od /MDd")
    set(COMMON_RELEASE "/O2 /MD /D NDEBUG")
endif()
set(CMAKE_CXX_FLAGS_DEBUG "${COMMON_DEBUG}")
set(CMAKE_CXX_FLAGS_RELEASE "${COMMON_RELEASE}")
set(CMAKE_C_FLAGS_DEBUG "${COMMON_DEBUG}")
set(CMAKE_C_FLAGS_RELEASE "${COMMON_RELEASE}")

# `递归包含工程定义 *.cmake文件`
macro(COMMON_PROJECT)
set(COMMON_PROJECT_FILTER "*.cmake")
foreach(basedir ${ARGV})
    file(GLOB_RECURSE COMMON_PROJECT_FILES "${basedir}/${COMMON_PROJECT_FILTER}")
    foreach(project_file ${COMMON_PROJECT_FILES})
        message(STATUS "project file found -- ${project_file}")
        include("${project_file}")
    endforeach()
endforeach()
endmacro(COMMON_PROJECT)

# `颜色回显`
function(CommonEcho)
    # ${ARGV}, ${ARGN}
    set(ECHO_WITH_COLOR_CMD "echo")
    set(ECHO_WITH_COLOR_CMD_DP "")
    if(UNIX OR CYGWIN OR APPLE)
        set(TAG_RED     "\\033[31;1m")
        set(TAG_GREEN   "\\033[32;1m")
        set(TAG_YELLOW  "\\033[33;1m")
        set(TAG_BLUE    "\\033[34;1m")
        set(TAG_PURPLE  "\\033[35;1m")
        set(TAG_CYAN    "\\033[36;1m")
        set(TAG_RESET   "\\033[;0m")
        set(ECHO_WITH_COLOR_CMD_DP "-e")
    elseif(WIN32)
        set(TAG_RED     "")
        set(TAG_GREEN   "")
        set(TAG_YELLOW  "")
        set(TAG_BLUE    "")
        set(TAG_PURPLE  "")
        set(TAG_CYAN    "")
        set(TAG_RESET   "")
    endif()

    set(ECHO_WITH_COLOR_PREFIX "")
    set(ECHO_WITH_COLOR_SUFFIX "")
    set(ECHO_WITH_COLOR_FLAG "false")
    foreach (msg IN LISTS ARGV)
        if ( "${msg}" STREQUAL "COLOR" )
            set(ECHO_WITH_COLOR_FLAG "true")
        elseif( "${ECHO_WITH_COLOR_FLAG}" STREQUAL "true" )
            set(ECHO_WITH_COLOR_FLAG "false")
            if ("${msg}" STREQUAL "RED")
                set(ECHO_WITH_COLOR_PREFIX "${TAG_RED}")
                set(ECHO_WITH_COLOR_SUFFIX "${TAG_RESET}")
            elseif ("${msg}" STREQUAL "GREEN")
                set(ECHO_WITH_COLOR_PREFIX "${TAG_GREEN}")
                set(ECHO_WITH_COLOR_SUFFIX "${TAG_RESET}")
            elseif ("${msg}" STREQUAL "YELLOW")
                set(ECHO_WITH_COLOR_PREFIX "${TAG_YELLOW}")
                set(ECHO_WITH_COLOR_SUFFIX "${TAG_RESET}")
            elseif ("${msg}" STREQUAL "BLUE")
                set(ECHO_WITH_COLOR_PREFIX "${TAG_BLUE}")
                set(ECHO_WITH_COLOR_SUFFIX "${TAG_RESET}")
            elseif ("${msg}" STREQUAL "PURPLE")
                set(ECHO_WITH_COLOR_PREFIX "${TAG_PURPLE}")
                set(ECHO_WITH_COLOR_SUFFIX "${TAG_RESET}")
            elseif ("${msg}" STREQUAL "CYAN")
                set(ECHO_WITH_COLOR_PREFIX "${TAG_CYAN}")
                set(ECHO_WITH_COLOR_SUFFIX "${TAG_RESET}")
            else ()
                message(WARNING "EchoWithColor ${msg} not supported.")
            endif()
        else()
            execute_process(COMMAND ${ECHO_WITH_COLOR_CMD} ${ECHO_WITH_COLOR_CMD_DP} "${ECHO_WITH_COLOR_PREFIX}${msg}${ECHO_WITH_COLOR_SUFFIX}")
        endif()
    endforeach()

endfunction(CommonEcho)

# `编译平台`
if ("${CMAKE_CXX_SIZEOF_DATA_PTR}" STREQUAL "4")
    CommonEcho(COLOR RED "-- platform ${CMAKE_CXX_PLATFORM_ID} 32")
elseif ("${CMAKE_CXX_SIZEOF_DATA_PTR}" STREQUAL "8")
    CommonEcho(COLOR RED "-- platform ${CMAKE_CXX_PLATFORM_ID} 64")
else()
    CommonEcho(COLOR RED "-- platform ${CMAKE_CXX_PLATFORM_ID} ??")
endif()
\end{lstlisting}

{gbase.cmake: }\par
\begin{lstlisting} [language=bash]
include("${CMAKE_SOURCE_DIR}/common.cmake")

# `包含头文件`
include_directories("${CMAKE_SOURCE_DIR}")

set(GBASE_DIR_CORE "${CMAKE_SOURCE_DIR}/core")
set(GBASE_DIR_DS "${CMAKE_SOURCE_DIR}/ds")
set(GBASE_DIR_NET "${CMAKE_SOURCE_DIR}/net")
set(GBASE_DIR_UTIL "${CMAKE_SOURCE_DIR}/util")
set(GBASE_DIR_TEST "${CMAKE_SOURCE_DIR}/test")

set(GBASE_LIB gbase)

# `链接选项`
set (GBASE_LIB_LINK ${COMMON_LINK_LIB})

# `编译lib的源文件`
aux_source_directory(${GBASE_DIR_CORE} GBASE_SOURCE)
aux_source_directory(${GBASE_DIR_DS} GBASE_SOURCE)
aux_source_directory(${GBASE_DIR_NET} GBASE_SOURCE)
aux_source_directory(${GBASE_DIR_UTIL} GBASE_SOURCE)
foreach(GBASE_SOURCE_FILE ${GBASE_SOURCE})
    CommonEcho(COLOR CYAN "===> source: ${GBASE_SOURCE_FILE}")
endforeach()

# `编译lib`
add_library(${GBASE_LIB} ${GBASE_SOURCE})

# `递归增加test`
file(GLOB GBASE_TEST_DIRS ${GBASE_DIR_TEST}/*test*)
foreach(GBASE_TEST_DIR ${GBASE_TEST_DIRS})
    CommonEcho(COLOR CYAN "===> directory: ${GBASE_TEST_DIR}")
    add_subdirectory(${GBASE_TEST_DIR})
endforeach()
\end{lstlisting}
}

\end {document}
