\begin{document}
\title {\ZHH \huge 常用bash shell整理}
\author {\small gaccob}
\date {\small 2014 年 1 月 26 日}
\maketitle

\section {\ZHH bash脚本的参数} {
    { \textcolor{blue}{手工的处理方式}: } \par
    \begin{itemize}
    \item {\small \$0: 指脚本自己, 例如test.sh. }
    \item {\small \$1: 第一个参数, 依次类推\$2, \$3 ... }
    \item {\small \$\#: 参数的个数. }
    \item {\small \$@: 所有的参数, 是一个数组. }
    \item {\small \$*: 所有的参数, 与\$@不一样的是它是一个字符串. }
    \end{itemize} \par
    { 一个简单的使用sample: }\par
    \gaccobsplitinv
    \begin{lstlisting}[language=bash]
    for arg in "$@"
    do
        echo $arg
    done
    \end{lstlisting}

    \gaccobsplitinv

    { \textcolor{blue}{getopts的处理方式}: }\par
    { (这里只讨论短选项, 长选项需要getopt, 会复杂一些). } \par
    \gaccobsplitinv
    \begin{lstlisting}[language=bash]
    while getopts "a:bc" arg
    do
        case $arg in
            a) echo -e "argument a: $OPTARG\n" ;;
            b) echo -e "argument b\n" ;;
            c) echo -e "argument c\n" ;;
            ?) echo -e "usage: ./test <-a **> <-b> <-c>" ;;
        esac
    done
    \end{lstlisting}
    { 上面这个sample中: }\par
    \begin{itemize}
    \item {\small "a:bc"代表了-a需要参数, b和c是无参数的选项. }
    \item {\small case ? 代表了任何不识别的选项. }
    \end{itemize}
}


\section {\ZHH 获取当前脚本目录} {
    \begin{lstlisting}[language=bash]
    function get_pwd()
    {
        echo "$( cd "$( dirname "$0" )" && pwd )";
    }
    \end{lstlisting}
}


\section {\ZHH 遍历嵌套文件夹} {
    \begin{lstlisting}[language=bash]
    function loop_dir()
    {
        dir="$1";
        for sub in $dir
        do
            if [ -d "$sub" ]; then
                loop_dir $sub;
            else
                # do something
            fi
        done
    }
    \end{lstlisting}
}


\section {\ZHH expect的典型用法} {
    {通过expect完成自动scp的过程参考: }\par
    \begin{lstlisting}[language=bash]
    function scp_target()
    {
        scp_src="$1";
        scp_dst="$2";
        scp_dst_passwd="$3";
        expect -c "set timeout -1;
            spawn scp -p -o StrictHostKeyChecking=no -r $scp_src $scp_dst;
            expect "*assword:*" { send $scp_dst_passwd\r\n; }; 
            expect eof {exit;}; "
    }
    \end{lstlisting}
}

\section {\ZHH 进程锁} {
    {通过flock实现进程锁, 尤其在执行shell脚本时比较实用, 具体可以参考man page. }\par
    {一个典型用法如下: }\par
    \begin{lstlisting}[language=bash]
    # `non-block (fail exit with code 1), exclusive lock`
    flock -xn /tmp/test.lock -c test.sh

    # `non-block, wait 5 seconds`
    flock -n -w 5 /tmp/test.lock -c test.sh
    \end{lstlisting}
}

\end{document}
