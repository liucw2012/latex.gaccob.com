\begin {document}

\title {\ZHH \huge 学习ejoy2d——sprite}
\author {\small gaccob}
\date {\small 2014 年 3 月 14 日}
\maketitle

\section {\ZHH sprite是什么} {
    { "sprite是ejoy2d中可以处理的基本图形对象, 每个 sprite 都是若干图元以树状组合起来的". 目前只有animation图元类型的sprite有孩子, 其他都是单个节点的存在. }\par
}

\section {\ZHH sprite属性} {

    { 结合github上的文档说明, 先看一下C kernel中sprite数据结构定义.}\par

\begin{lstlisting}[language=C]
// file:  lib/spritepack.h
struct sprite_trans {
    struct matrix * mat;
    uint32_t color;
    uint32_t additive;
    int program;
};

// file:  lib/sprite.h
struct sprite {

    // `\color{gray} 父节点, 与children节点一起, 维系了树状结构`
    // `\color{gray} lua接口: sprite.has\_parent(只读), sprite.parent\_name(只读)`
    struct sprite * parent;

    // `\color{gray} 图元类型`
    uint16_t type;

    // `\color{gray} 唯一id, 因为是数组, 所以不建议散的太开`
    uint16_t id;

    // `\color{gray} t.mat 渲染时的变换矩阵, 运行期, 默认单位矩阵`
    // `\color{gray} t.color 渲染时的混合颜色, ARGB32, 默认为0xFFFFFFFF, 作用域是整个子树, 最常见的是做alpha半透明效果, 例如0x80FFFFFF就是50\%的半透明`
    // `\color{gray} t.additive 渲染时的叠加颜色, RGB24, 默认为0, 作用域是整个子树`
    // `\color{gray} t.program 指定shader`
    // `\color{gray} Lua API: sprite.matrix(读写), sprite.color(读写), sprite.additive(读写), sprite.program(只读)`
    struct sprite_trans t;

    // `\color{gray} 5种基本图元`
    union {
        struct pack_animation *ani;
        struct pack_picture *pic;
        struct pack_polygon *poly;
        struct pack_label *label;
        struct pack_pannel *pannel;
        struct matrix *mat;
    } s;

    // `\color{gray} 只读anchor的特殊属性, 返回上一次这个anchor对象最终渲染的世界矩阵`
    // `\color{gray} anchor.visible=false, 当不可显时,引擎不计算 world matrix`
    // `\color{gray} Lua API: sprite.wolrd\_matrix(只读)`
    struct matrix mat;

    // `\color{gray} 总帧数 与 开始帧数`
    // `\color{gray} Lua API: sprite.frame\_count(只读)`
    int start_frame;
    int total_frame;

    // `\color{gray} 当前帧号`
    // `\color{gray} Lua API: sprite.frame(读写)`
    int frame;

    // `\color{gray} 如果设置为false, 则整个子树不显示`
    // `\color{gray} Lua API: sprite.visible(读写)`
    bool visible;

    // `\color{gray} 对象是否截获 test 调用, 多用于 UI 控制`
    // `\color{gray} Lua API: sprite.message(读写)`
    bool message;

    // `\color{gray} Lua API: sprite.name(只读)`
    const char *name;

    union {
        struct sprite * children[1];

        // `\color{gray} label的文字`
        // `\color{gray} Lua API: sprite.text(读写)`
        const char * text;

        // `\color{gray} panel是否有scissor`
        // `\color{gray} Lua API: sprite.scissor(只读)`
        int scissor;
    } data;
};

// `\color{gray} 上面注释中提到的Lua API基本都包含在了getter\&setter中`
// file:  lib/lsprite.c
static void
lgetter(lua_State *L) {
    luaL_Reg l[] = {
        {"frame", lgetframe},
        {"frame_count", lgettotalframe },
        {"visible", lgetvisible },
        {"name", lgetname },
        {"text", lgettext},
        {"color", lgetcolor },
        {"additive", lgetadditive },
        {"message", lgetmessage },
        {"matrix", lgetmat },
        {"world_matrix", lgetwmat },
        {"parent_name", lgetparentname },
        {"has_parent", lhasparent },
        {NULL, NULL},
    };
    luaL_newlib(L,l);
}

static void
lsetter(lua_State *L) {
    luaL_Reg l[] = {
        {"frame", lsetframe},
        {"action", lsetaction},
        {"visible", lsetvisible},
        {"matrix" , lsetmat},
        {"text", lsettext},
        {"color", lsetcolor},
        {"additive", lsetadditive },
        {"message", lsetmessage },
        {"program", lsetprogram },
        {"scissor", lsetscissor },
        {NULL, NULL},
    };
    luaL_newlib(L,l);
}
\end{lstlisting}

    {上层Lua中, 依据setter\&getter接口, 设置sprite的metatable. }\par

\begin{lstlisting}[language=lua]
// file:  ejoy2d/sprite.lua
function sprite_meta.__index(spr, key)
    if method[key] then
        return method[key]
    end
    local getter = get[key]
    if getter then
        return getter(spr)
    end
    local child = fetch(spr, key)

    if child then
        return child
    else
        print("Unsupport get " ..  key)
        return nil
    end
end

function sprite_meta.__newindex(spr, key, v)
    local setter = set[key]
    if setter then
        setter(spr, v)
        return
    end
    assert(debug.getmetatable(v) == sprite_meta, "Need a sprite")
    method.mount(spr, key, v)
end
\end{lstlisting}

}

\section {\ZHH sprite.new} {
    {Lua接口sprite.new的实现如下: }\par
\begin{lstlisting}[language=lua]
-- file: ejoy2d/sprite.lua
function sprite.new(packname, name)

    -- `\color{gray} 这里的资源是预先已经初始化过的`
    local pack, id = pack.query(packname, name)

    -- `\color{gray} 调用C接口实现new`
    local cobj = c.new(pack,id)

    -- `\color{gray} 设置meta`
    if cobj then
        return debug.setmetatable(cobj, sprite_meta)
    end
end

-- file: ejoy2d/spritepack.lua
-- `\color{gray} 从已经import的资源中查询spritepack数据, name可以是名字或者id`
function spritepack.query( packname, name )
    local p = assert(pack_pool[packname], "Load package first")
    local id
    if type(name) == "number" then
        id = name
    else
        id = p.export[name]
    end
    if not id then
        error(string.format("'%s' is not exist in package %s", name, packname))
    end
    return p.cobj, id
end
\end{lstlisting}

    {相关的C代码如下:} \par
\begin{lstlisting}[language=C]
// file: lib/lsprite.c
// `\color{gray} 输入: userdata spritepack`
//      `\color{gray} number id`
// `\color{gray} 输出: userdata sprite`
static int
lnew(lua_State *L) {
    struct sprite_pack * pack = (struct sprite_pack *)lua_touserdata(L, 1);
    if (pack == NULL) {
        return luaL_error(L, "Need a sprite pack");
    }
    int id = (int)luaL_checkinteger(L, 2);
    struct sprite * s = newsprite(L, pack, id);
    if (s) {
        return 1;
    }
    return 0;
}

// `\color{gray} 图元类型的sprite new`
static struct sprite *
newsprite(lua_State *L, struct sprite_pack *pack, int id) {
    // `\color{gray} anchor类型单独new`
    if (id == ANCHOR_ID) {
        return newanchor(L);
    }

    // `\color{gray} 计算sprite size, 从Lua VM中分配内存`
    int sz = sprite_size(pack, id);
    if (sz == 0) {
        return NULL;
    }
    struct sprite * s = (struct sprite *)lua_newuserdata(L, sz);

    // `\color{gray} 从spritepack初始化sprite数据`
    sprite_init(s, pack, id, sz);

    // `\color{gray} 递归载入孩子sprite(只有animation会触发)`
    int i;
    for (i=0;;i++) {

        // `\color{gray} 引用的component`
        int childid = sprite_component(s, i);
        if (childid < 0)
            break;

        if (i==0) {
            lua_newtable(L);
            lua_pushvalue(L,-1);
            lua_setuservalue(L, -3);    // set uservalue for sprite
        }

        // `\color{gray} 初始化child`
        struct sprite *c = newsprite(L, pack, childid);

        // `\color{gray} 设置child名字`
        c->name = sprite_childname(s, i);

        // `\color{gray} 挂载child, 组织成树状结构`
        sprite_mount(s, i, c);

        // `\color{gray} 如果touchable, sprite->message设置为true`
        update_message(c, pack, id, i, s->frame);

        // `\color{gray} todo:`
        if (c) {
            lua_rawseti(L, -2, i+1);
        }
    }
    if (i>0) {
        lua_pop(L,1);
    }
    return s;
}

// `\color{gray} anchor类型的sprite, 没有spritepack数据`
static struct sprite *
newanchor(lua_State *L) {
    // `\color{gray} 比常规sprite多了个matrix`
    // `\color{gray} 此处代码加到sprite\_size()更好看一些, 这里为了少一次函数调用?`
    int sz = sizeof(struct sprite) + sizeof(struct matrix);
    struct sprite * s = (struct sprite *)lua_newuserdata(L, sz);

    s->parent = NULL;
    s->t.mat = NULL;
    s->t.color = 0xffffffff;
    s->t.additive = 0;
    s->t.program = PROGRAM_DEFAULT;
    s->message = false;

    // `\color{gray} 默认不可见`
    s->visible = false;

    s->name = NULL;

    // `\color{gray} 所有的anchor类型都是一个默认id`
    s->id = ANCHOR_ID;

    s->type = TYPE_ANCHOR;

    // `\color{gray} anchor类型用到的matrix, 不像其他图元类型sprite数据在pack包中,`
    // `\color{gray} 所以这里要自己分配内存, 在sprite对象之后`
    s->s.mat = (struct matrix *)(s+1);
    matrix_identity(s->s.mat);

    return s;
}

// file: lib/sprite.c
// `\color{gray} 从spritepack数据初始化sprite对象`
void
sprite_init(struct sprite * s, struct sprite_pack * pack, int id, int sz) {
    if (id < 0 || id >= pack->n)
        return;
    s->parent = NULL;
    s->t.mat = NULL;
    s->t.color = 0xffffffff;
    s->t.additive = 0;
    s->t.program = PROGRAM_DEFAULT;
    s->message = false;
    s->visible = true;
    s->name = NULL;
    s->id = id;
    s->type = pack->type[id];

    // `\color{gray} animation类型单独处理, 设置frame和action`
    if (s->type == TYPE_ANIMATION) {
        struct pack_animation * ani = (struct pack_animation *)pack->data[id];
        s->s.ani = ani;
        s->frame = 0;

        // `\color{gray} 默认只有一个action, 传入NULL`
        sprite_action(s, NULL);

        // `\color{gray} 这里时保证分配的内存足够用`
        // `\color{gray} animation包括孩子节点, 多出来n-1个child指针`
        // `\color{gray} child默认都是NULL`
        int i;
        int n = ani->component_number;
        assert(sz >= sizeof(struct sprite) + (n - 1) * sizeof(struct sprite *));
        for (i=0; i<n ;i++) {
            s->data.children[i] = NULL;
        }
    }
    // `\color{gray} 其他类型`
    else {
        // `\color{gray} 因为是union, data是void*, 指定任意其他图元类型都一样`
        s->s.pic = (struct pack_picture *)pack->data[id];

        // `\color{gray} 非animation无需frame`
        s->start_frame = 0;
        s->total_frame = 0;
        s->frame = 0;

        // `\color{gray} API sprite.text设置, 默认NULL`
        s->data.text = NULL;

        // `\color{gray} 减掉一个指针大小, 是因为无需计算animation child, 还是代码没修改?`
        assert(sz >= sizeof(struct sprite) - sizeof(struct sprite *));

        // `\color{gray} pannel类型需要设置scissor`
        if (s->type == TYPE_PANNEL) {
            struct pack_pannel * pp = (struct pack_pannel *)pack->data[id];
            s->data.scissor = pp->scissor;
        }
    }
}

\end{lstlisting}
}

\end{document}

