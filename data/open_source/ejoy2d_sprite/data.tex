\begin {document}

\title {\ZHH \huge 学习ejoy2d——sprite}
\author {\small gaccob}
\date {\small 2014 年 3 月 12 日}
\maketitle

\section {\ZHH sprite\_pack} {

    {在学习sprite之前, 先了解一下sprite\_pack, 它的主要作用是在运行时导入二进制的资源文件, 导入的数据就存储在sprite\_pack数据结构中. (根据文档所述, 有\href{https://github.com/robinxb/flash-parser}{开源工具}支持从flash中导出二进制资源). "这只是方便开发,在产品发行时不应该这样做". 从版本的安全性来说, 资源还需要做加密, 防止破解. }\par

    {sprite目前支持5种图元:}\par
\begin{lstlisting}[language=C]
#define TYPE_EMPTY 0
#define TYPE_PICTURE 1
#define TYPE_ANIMATION 2
#define TYPE_POLYGON 3
#define TYPE_LABEL 4
#define TYPE_PANNEL 5
// `\color{gray} anchor是一个比较特殊的类型, 它没有资源数据, 仅提供了一个锚点, 可以挂载其他sprite`
#define TYPE_ANCHOR 6

// `\color{gray} TYPE\_PICTURE 四边形`
struct pack_picture {
    int n;
    struct pack_quad rect[1];
};

// `\color{gray} TYPE\_ANIMATION 动画`
struct pack_animation {
    int frame_number;
    int action_number;
    int component_number;
    struct pack_frame *frame;
    struct pack_action *action;
    struct pack_component component[1];
};

// `\color{gray} TYPE\_POLYGON 多边形`
struct pack_poly {
    int texid;
    int n;
    uint16_t *texture_coord;
    int32_t *screen_coord;
};

// `\color{gray} TYPE\_LABEL 文字框`
struct pack_label {
    uint32_t color;
    int width;
    int height;
    int align;
    int size;
    int edge;
    int max_width;
};

// `\color{gray} TYPE\_PANNEL 面板`
struct pack_pannel {
    int width;
    int height;
    int scissor;
};

// `\color{gray} component可以是任意sprite, 但是要自己保证不能成环`
struct pack_component {
    int id;
    const char *name;
};

// `\color{gray} 这就是导入的lua描述文件的C数据结构`
// `\color{gray} data数组的下标是sprite的id, 也就是component引用的id.`
struct sprite_pack {
    int n;
    uint8_t * type;
    void ** data;
    int tex[1];
};
\end{lstlisting}

    {spritepack中最重要的就是limport()函数, 它导入二进制资源到C数据结构中, 具体逻辑看源码:}\par
\begin{lstlisting}[language=C]
// `\color{gray} lua的输入参数格式:`
// `\color{gray}    number: texture id |  table: texture id table`
// `\color{gray}    number: max id, pack对象的最大id`
// `\color{gray}    number: max userdata size, 分配器的buffer size`
// `\color{gray}    string: data | lightuserdate: data`
// `\color{gray}                 | number: data size`
static int
limport(lua_State *L) {
    ...

    // `\color{gray} 一个简单的分配器, max size来自lua的传入参数`
    struct import_alloc alloc;
    alloc.L = L;
    alloc.buffer = (char *)lua_newuserdata(L, size);
    alloc.cap = size;
   
    // `\color{gray} 从分配器中(userdata)分配sprite\_pack, 并初始化`
    struct sprite_pack *pack = (struct sprite_pack *)ialloc(&alloc, sizeof(*pack) + (tex - 1) * sizeof(int));
    pack->n = max_id + 1;

    // `\color{gray} 分配type, 4字节对齐(type是uint8)`
    int align_n = (pack->n + 3) & ~3;
    pack->type = (uint8_t *)ialloc(&alloc, align_n * sizeof(uint8_t));
    memset(pack->type, 0, align_n * sizeof(uint8_t));
    
    // `\color{gray} 分配data指针`
    pack->data = (void **)ialloc(&alloc, pack->n * sizeof(void*));
    memset(pack->data, 0, pack->n * sizeof(void*));

    // `\color{gray} 导入texure id`
    if (lua_istable(L,1)) {
        int i;
        for (i=0; i<tex; i++) {
            lua_rawgeti(L,1,i+1);
            pack->tex[i] = (int)luaL_checkinteger(L, -1);
            lua_pop(L,1);
        }
    } else {
        pack->tex[0] = (int)lua_tointeger(L,1);
    }

    // `\color{gray} 构造一个输入数据流(指向目标资源数据), 方便后续导入`
    struct import_stream is;
    is.alloc = &alloc;
    is.pack = pack;
    is.current_id = -1;
    if (lua_isstring(L,4)) {
        is.stream = lua_tolstring(L, 4, &is.size);
    } else {
        is.stream = (const char *)lua_touserdata(L, 4);
        if (is.stream == NULL) {
            return luaL_error(L, "Need const char *");
        }
        is.size = luaL_checkinteger(L, 5);
    }

    // `\color{gray} 依次从数据流中导入数据`
    while (is.size != 0) {
        import_sprite(&is);
    }

    return 1;
}
\end{lstlisting}

    {在import\_sprite()函数中, 依次import 2字节的sprite id, 1字节的type, 然后就根据不同的type分别import不同的sprite图元. 这些图元数据内存都是从import\_stream中的分配器中分配(limport()中定义). }\par
    {具体的过程可以参阅源码, 逻辑都比较简单, 唯一需要注意的是anchor类型的sprite id是一个特殊的默认值ANCHOR\_ID(0xffff), 区别于其他常规sprite. }\par
}

\section {\ZHH sprite} {

    { sprite是ejoy2d中最复杂的数据结构, "每个 sprite 都是若干图元以树状组合起来的".}\par

\begin{lstlisting}[language=C]
struct sprite {
    struct sprite * parent;
    uint16_t type;

    // `\color{gray} 唯一id, 因为是数组, 所以不建议散的太开`
    uint16_t id;

    struct sprite_trans t;

    // `\color{gray} 5种基本图元`
    union {
        struct pack_animation *ani;
        struct pack_picture *pic;
        struct pack_polygon *poly;
        struct pack_label *label;
        struct pack_pannel *pannel;
        struct matrix *mat;
    } s;

    // 渲染时的附加矩阵
    struct matrix mat;

    int start_frame;
    int total_frame;
    int frame;
    bool visible;
    bool message;
    const char *name;	// name for parent
    union {
        struct sprite * children[1];
        const char * text;
        int scissor;
    } data;
};
\end{lstlisting}

}

\end{document}

