\begin {document}

\title {\ZHH \huge 学习ejoy2d——sprite}
\author {\small gaccob}
\date {\small 2014 年 3 月 14 日}
\maketitle

\section {\ZHH sprite是什么} {
    { "sprite是ejoy2d中可以处理的基本图形对象, 每个 sprite 都是若干图元以树状组合起来的".}\par
}

\section {\ZHH sprite属性} {

    { 结合github上的文档说明, 先看一下C kernel中sprite数据结构定义.}\par

\begin{lstlisting}[language=C]
// file:  lib/spritepack.h
struct sprite_trans {
    struct matrix * mat;
    uint32_t color;
    uint32_t additive;
    int program;
};

// file:  lib/sprite.h
struct sprite {

    // `\color{gray} 父节点, 与children节点一起, 维系了树状结构`
    // `\color{gray} lua接口: sprite.has\_parent(只读), sprite.parent\_name(只读)`
    struct sprite * parent;

    // `\color{gray} 图元类型`
    uint16_t type;

    // `\color{gray} 唯一id, 因为是数组, 所以不建议散的太开`
    uint16_t id;

    // `\color{gray} t.mat 渲染时的变换矩阵, 运行期, 默认单位矩阵`
    // `\color{gray} t.color 渲染时的混合颜色, ARGB32, 默认为0xFFFFFFFF, 作用域是整个子树, 最常见的是做alpha半透明效果, 例如0x80FFFFFF就是50\%的半透明`
    // `\color{gray} t.additive 渲染时的叠加颜色, RGB24, 默认为0, 作用域是整个子树`
    // `\color{gray} t.program 指定shader`
    // `\color{gray} Lua API: sprite.matrix(读写), sprite.color(读写), sprite.additive(读写), sprite.program(只读)`
    struct sprite_trans t;

    // `\color{gray} 5种基本图元`
    union {
        struct pack_animation *ani;
        struct pack_picture *pic;
        struct pack_polygon *poly;
        struct pack_label *label;
        struct pack_pannel *pannel;
        struct matrix *mat;
    } s;

    // `\color{gray} 只读anchor的特殊属性, 返回上一次这个anchor对象最终渲染的世界矩阵`
    // `\color{gray} anchor.visible=false, 当不可显时,引擎不计算 world matrix`
    // `\color{gray} Lua API: sprite.wolrd\_matrix(只读)`
    struct matrix mat;

    // `\color{gray} 总帧数 与 开始帧数`
    // `\color{gray} Lua API: sprite.frame\_count(只读)`
    int start_frame;
    int total_frame;

    // `\color{gray} 当前帧号`
    // `\color{gray} Lua API: sprite.frame(读写)`
    int frame;

    // `\color{gray} 如果设置为false, 则整个子树不显示`
    // `\color{gray} Lua API: sprite.visible(读写)`
    bool visible;

    // `\color{gray} 对象是否截获 test 调用, 多用于 UI 控制`
    // `\color{gray} Lua API: sprite.message(读写)`
    bool message;

    // `\color{gray} Lua API: sprite.name(只读)`
    const char *name;

    union {
        struct sprite * children[1];

        // `\color{gray} label的文字`
        // `\color{gray} Lua API: sprite.text(读写)`
        const char * text;

        // `\color{gray} panel是否有scissor`
        // `\color{gray} Lua API: sprite.scissor(只读)`
        int scissor;
    } data;
};

// `\color{gray} 上面注释中提到的Lua API基本都包含在了getter\&setter中`
// file:  lib/lsprite.c
static void
lgetter(lua_State *L) {
	luaL_Reg l[] = {
		{"frame", lgetframe},
		{"frame_count", lgettotalframe },
		{"visible", lgetvisible },
		{"name", lgetname },
		{"text", lgettext},
		{"color", lgetcolor },
		{"additive", lgetadditive },
		{"message", lgetmessage },
		{"matrix", lgetmat },
		{"world_matrix", lgetwmat },
		{"parent_name", lgetparentname },
		{"has_parent", lhasparent },
		{NULL, NULL},
	};
	luaL_newlib(L,l);
}

static void
lsetter(lua_State *L) {
	luaL_Reg l[] = {
		{"frame", lsetframe},
		{"action", lsetaction},
		{"visible", lsetvisible},
		{"matrix" , lsetmat},
		{"text", lsettext},
		{"color", lsetcolor},
		{"additive", lsetadditive },
		{"message", lsetmessage },
		{"program", lsetprogram },
		{"scissor", lsetscissor },
		{NULL, NULL},
	};
	luaL_newlib(L,l);
}
\end{lstlisting}

    {上层Lua中, 依据setter\&getter接口, 设置sprite的metatable. }\par

\begin{lstlisting}[language=lua]
// file:  ejoy2d/sprite.lua
function sprite_meta.__index(spr, key)
	if method[key] then
		return method[key]
	end
	local getter = get[key]
	if getter then
		return getter(spr)
	end
	local child = fetch(spr, key)

	if child then
		return child
	else
		print("Unsupport get " ..  key)
		return nil
	end
end

function sprite_meta.__newindex(spr, key, v)
	local setter = set[key]
	if setter then
		setter(spr, v)
		return
	end
	assert(debug.getmetatable(v) == sprite_meta, "Need a sprite")
	method.mount(spr, key, v)
end
\end{lstlisting}

}

\section {\ZHH sprite方法} {

    {看完了C kernel中sprite的数据结构及Lua metatable(属性), 下面来看一下sprite的方法.}\par

}

\end{document}

