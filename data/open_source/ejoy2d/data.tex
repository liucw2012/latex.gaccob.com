\begin {document}

\title {\ZHH \huge ejoy2d源码学习}
\author {\small gaccob}
\date {\small 2014 年 1 月 25 日}
\maketitle

\vspace {10pt}
\section{\huge \ZHH ejoy2d.matrix} {
    { 在学习这一段代码之前, 先引入一些{\ZHH{矩阵的基础理论}}. }

    \begin {itemize}

    \item {
        2d图形的变换, 可以用一个6元组$\{a, b, c, d, t_x, t_y\}$来表示, 其中, $a$和$d$表示缩放, $b$和$c$表示旋转, $t_x$和$t_y$表示位移, 这样就能得到一个基础的2d矩阵公式:
        \begin{equation}\label{raw1}
            x = ax + cy + t_x
        \end{equation}
        \begin{equation}\label{raw2}
            y = bx + dy + t_y
        \end{equation}
    }

    \item {
        为了把二维图形的变化统一在一个坐标系里, 引入齐次坐标的概念, 即把一个图形用一个三维矩阵表示, 其中第三列总是$(0,0,1)$, 用来作为坐标系的标准, 这样, 变换6元组就可以用一个3*3的矩阵来表示:
         \begin{equation}\label{matrix}
            M =
            \begin{bmatrix}
                a & b & 0 \\
                c & d & 0 \\
                t_x & t_y & 1
            \end{bmatrix}
        \end{equation}
        这样, 我们就可以得到一个矩阵变换公式:
        \begin{equation}\label{transform}
            \begin{bmatrix}
                x & y & 1
            \end{bmatrix}
            * M =
            \begin{bmatrix}
                ax + cy + t_x & bx + dy + t_y & 1
            \end{bmatrix}
        \end{equation}
    }

    \item {
        {\ZHH{平移}}: 当$a = d = 1$, $c = d = 0$时, 则\eqref{transform}退化为:
        \begin{equation}\label{transform_shift}
            \begin{bmatrix}
                x & y & 1
            \end{bmatrix}
            * M =
            \begin{bmatrix}
                x + t_x & y + t_y & 1
            \end{bmatrix}
        \end{equation}
    }

    \item {
        {\ZHH{缩放}}: 当$b = c = t_x = t_y = 0$时, 则\eqref{transform}退化为:
        \begin{equation}\label{transform_scale}
            \begin{bmatrix}
                x & y & 1
            \end{bmatrix}
            * M =
            \begin{bmatrix}
                ax & dy & 1
            \end{bmatrix}
        \end{equation}
    }

    \item {
        {\ZHH{旋转}}: 当$t_x = t_y = 0$, $a = cos\theta$, $b = sin\theta$, $c = -sin\theta$, $d = cos\theta$时, 则\eqref{transform}退化为:
        \begin{equation}\label{transform_rotate}
            \begin{bmatrix}
                x & y & 1
            \end{bmatrix}
            * M =
            \begin{bmatrix}
                x*cos\theta - y*sin\theta & x*sin\theta + y*cos\theta & 1
            \end{bmatrix}
         \end{equation}
    }

    \item {
        {\ZHH{归一化}}: 当$a = d = 1$, $c = d = t_x = t_y = 0$时, 则M退化为归一化矩阵:
        \begin{equation}\label{identity}
            E =
            \begin{bmatrix}
                1 & 0 & 0\\
                0 & 1 & 0\\
                0 & 0 & 1
            \end{bmatrix}
        \end{equation}
    }

    \item {
        {\ZHH{逆矩阵}}: 所谓逆矩阵, 就是对于矩阵$A$, 存在矩阵$B$, $A * B = B * A = E$, 其中, $E$是归一化矩阵, 那么$A$和$B$就互为各自的逆矩阵. 对于M而言, 它的逆矩阵可以计算得到:
        \begin{equation}\label{inverse}
            N =
            \begin{bmatrix}
                \frac{d}{ad - bc}           &   -\frac{b}{ad - bc}          &   0\\
                -\frac{c}{ad - bc}          &   \frac{a}{ad-bc}             &   0\\
                \frac{ct_y - dt_x}{ad - bc} &  \frac{bt_x - at_y}{ad - bc}  &   1\\
            \end{bmatrix}
        \end{equation}
        \begin{equation}
            M * N = N * M = E
        \end{equation}
    }
    \end {itemize}

    \gaccobsplit

    { 简单回顾一下理论, 下面就可以开始ejoy2d的{\ZHH{矩阵源码分析}}, lib/matrix.c和lib/matrix.h封装了矩阵的操作, lib/lmatrix.c和lib/lmatrix.h是lua调用c的接口.}

    \begin {enumerate}

    \gaccobsplitinv

    \item { 6元组表示的{\ZHH{矩阵(matrix)}}:
        \begin{lstlisting}[language={[ANSI]C}]
struct matrix {
int m[6];
};
        \end{lstlisting}
    }

    \gaccobsplitinv

    \item { {\ZHH{矩阵归一化(matrix identity)}}.
        \begin{lstlisting}[language={[ANSI]C}]
matrix_identity(struct matrix *mm) {
    int *mat = mm->m;
    mat[0] = 1024;
    mat[1] = 0;
    mat[2] = 0;
    mat[3] = 1024;
    mat[4] = 0;
    mat[5] = 0;
}
        \end{lstlisting}
        与\eqref{matrix}中不一样的是, 为了避免浮点运算, 这里以1024为基处理, 来保持整数运算, 实际上, 现在的变换矩阵\eqref{matrix}演变为:
        \begin{equation}\label{matrix_ejoy2d}
            M_{ejoy} =
            M *
            \begin{bmatrix}
                1024    &   0       &   0\\
                0       &   1024    &   0\\
                0       &   0       &   1024
            \end{bmatrix}
            =
            \begin{bmatrix}
                1024 * a    &   b           & 0 \\
                c           &   1024 * d    & 0 \\
                t_x         &   t_y         & 1024
            \end{bmatrix}
        \end{equation}
    }

    \gaccobsplitinv

    \item { {\ZHH{矩阵的乘法(matrix multiply)}}.
        \begin{lstlisting}[language={[ANSI]C}]
static inline void
matrix_mul(struct matrix *mm, const struct matrix *mm1, const struct matrix *mm2) {
    int *m = mm->m;
    const int *m1 = mm1->m;
    const int *m2 = mm2->m;
    m[0] = (m1[0] * m2[0] + m1[1] * m2[2]) /1024;
    m[1] = (m1[0] * m2[1] + m1[1] * m2[3]) /1024;
    m[2] = (m1[2] * m2[0] + m1[3] * m2[2]) /1024;
    m[3] = (m1[2] * m2[1] + m1[3] * m2[3]) /1024;
    m[4] = (m1[4] * m2[0] + m1[5] * m2[2]) /1024 + m2[4];
    m[5] = (m1[4] * m2[1] + m1[5] * m2[3]) /1024 + m2[5];
}
        \end{lstlisting}
        这里就是3*3的矩阵乘法, 唯一做了变化的就是在乘完之后除1024来保持scale.
    }

    \gaccobsplitinv

    \item { {\ZHH{矩阵的逆运算(matrix inverse)}}.
        \begin{lstlisting}[language={[ANSI]C}]
int
matrix_inverse(const struct matrix *mm, struct matrix *mo) {
    const int *m = mm->m;
    int *o = mo->m;
    if (m[1] == 0 && m[2] == 0) {
        return _inverse_scale(m,o);
    }
    if (m[0] == 0 && m[3] == 0) {
        return _inverse_rot(m,o);
    }
    int t = m[0] * m[3] - m[1] * m[2] ;
    if (t == 0)
        return 1;
    o[0] = (int32_t)((int64_t)m[3] * (1024 * 1024) / t);
    o[1] = (int32_t)(- (int64_t)m[1] * (1024 * 1024) / t);
    o[2] = (int32_t)(- (int64_t)m[2] * (1024 * 1024) / t);
    o[3] = (int32_t)((int64_t)m[0] * (1024 * 1024) / t);
    o[4] = - (m[4] * o[0] + m[5] * o[2]) / 1024;
    o[5] = - (m[4] * o[1] + m[5] * o[3]) / 1024;
    return 0;
}
        \end{lstlisting}
        这里有几点需要注意的:
        \begin{itemize}
        \item { 因为归一化时做了1024的线性变换, 所以这里有1024的参数.}
        \item { 当 $b = c = 0$ 时, 退化为线性变换\_inverse\_scale, 这里做了简化处理提高效率. }
        \item { 当 $a = d = 0$ 时, 退化为旋转变换\_inverse\_rot, 这里同样也做了简化处理提高效率. }
        \end{itemize}
    }

    \gaccobsplitinv

    \item { {\ZHH{矩阵的缩放(matrix scale)}}.
        \begin{lstlisting}[language={[ANSI]C}]
static inline void
scale_mat(int *m, int sx, int sy) {
    if (sx != 1024) {
        m[0] = m[0] * sx / 1024;
        m[2] = m[2] * sx / 1024;
        m[4] = m[4] * sx / 1024;
    }
    if (sy != 1024) {
        m[1] = m[1] * sy / 1024;
        m[3] = m[3] * sy / 1024;
        m[5] = m[5] * sy / 1024;
    }
}
        \end{lstlisting}
        我们在\eqref{transform_scale}中说到, 当$M$退化为只有$a$和$d$时, 可以看做一个缩放(scale). 这里的$a$对应就是代码中的$sx$, $b$对应的就是代码中的$sy$. 此时的\eqref{transform}演进成如下的形式:
        \begin{equation}
            M *
            \begin{bmatrix}
                sx  &   0   &   0\\
                0   &   sy  &   0\\
                0   &   0   &   1
            \end{bmatrix}\\
            =
            \begin{bmatrix}
                sx*a    &   sy*b    &   0\\
                sx*c    &   sy*d    &   0\\
                sx*t_x  &   sy*t_t  &   1
            \end{bmatrix}\\
        \end{equation}
    }

    \gaccobsplitinv

    \item { {\ZHH {矩阵的旋转(matrix rotate)}}
        \begin{lstlisting}[language={[ANSI]C}]
static inline void
rot_mat(int *m, int d) {
    if (d==0)
        return;
    int cosd = icosd(d);
    int sind = isind(d);

    int m0_cosd = m[0] * cosd;
    int m0_sind = m[0] * sind;
    int m1_cosd = m[1] * cosd;
    int m1_sind = m[1] * sind;
    int m2_cosd = m[2] * cosd;
    int m2_sind = m[2] * sind;
    int m3_cosd = m[3] * cosd;
    int m3_sind = m[3] * sind;
    int m4_cosd = m[4] * cosd;
    int m4_sind = m[4] * sind;
    int m5_cosd = m[5] * cosd;
    int m5_sind = m[5] * sind;

    m[0] = (m0_cosd - m1_sind) /1024;
    m[1] = (m0_sind + m1_cosd) /1024;
    m[2] = (m2_cosd - m3_sind) /1024;
    m[3] = (m2_sind + m3_cosd) /1024;
    m[4] = (m4_cosd - m5_sind) /1024;
    m[5] = (m4_sind + m5_cosd) /1024;
}
        \end{lstlisting}

        在\eqref{transform_rotate}这种情况下, $M$退化为旋转矩阵, 此处代码的原理也很容易理解:
        \begin{equation}
            M *
            \begin{bmatrix}
                cos\theta  &  sin\theta  &   0\\
                -sin\theta &  -cos\theta &   0\\
                0          &  0          &   1
            \end{bmatrix}\\
            =
            \begin{bmatrix}
                cos\theta*a - sin\theta*b     & sin\theta*a + cos\theta*b     & 0\\
                cos\theta*c - sin\theta*d     & sin\theta*c + cos\theta*d     & 0\\
                cos\theta*t_x - sin\theta*t_y & sin\theta*t_x + cos\theta*t_y & 1
            \end{bmatrix}
        \end{equation}

        代码中的$d$是经过变换的角度, 这个在lua的接口中有体现, 源码截取如下:
        \begin{lstlisting}[language={[ANSI]C}]
double r = luaL_checknumber(L,2);
matrix_rot(m, r * (1024.0 / 360.0));
        \end{lstlisting}

        代码中对角度的计算做了一个$cos$表, $cos$和$sin$的计算都是通过查表来得到的, 代码中做$sin$计算时的64, 是$sin\theta = cos(90 - \theta)$经过1024换算后得到的.
        \begin{lstlisting}[language={[ANSI]C}]
static inline int
icost(int dd) {
    static int t[256] = {
    ...
    };
    if (dd < 0) {
        dd = 256 - (-dd % 256);
    } else {
        dd %= 256;
    }

    return t[dd];
}

static inline int
icosd(int d) {
    int dd = d/4;
    return icost(dd);
}

static inline int
isind(int d) {
    int dd = 64 - d/4;
    return icost(dd);
}
        \end{lstlisting}
    }

    \gaccobsplitinv

    \item{ {\ZHH{矩阵的SRT变换}}\par
        ejoy2d.matrix中用一个srt(scale, rotate, translate)结构体来封装了变换矩阵, 对矩阵的srt操作依次就是:scale\_mat, rot\_mat以及translate(直接增加$M$中的$t_x$和$t_y$).
        \begin{lstlisting}[language=C]
struct srt {
	int offx;
	int offy;
	int scalex;
	int scaley;
	int rot;
};

void
matrix_srt(struct matrix *mm, const struct srt *srt) {
	scale_mat(mm->m, srt->scalex, srt->scaley);
	rot_mat(mm->m, srt->rot);
	mm->m[4] += srt->offx;
	mm->m[5] += srt->offy;
}


void
matrix_sr(struct matrix *mat, int sx, int sy, int d) {
	int *m = mat->m;
	int cosd = icosd(d);
	int sind = isind(d);

	int m0_cosd = sx * cosd;
	int m0_sind = sx * sind;
	int m3_cosd = sy * cosd;
	int m3_sind = sy * sind;

	m[0] = m0_cosd /1024;
	m[1] = m0_sind /1024;
	m[2] = -m3_sind /1024;
	m[3] = m3_cosd /1024;
}
        \end{lstlisting}
        其中函数matrix\_sr()是$b = c = 0$后的简化演进, 下面的公式省略了ejoy2d中1024的scale:
        \begin{equation}
        \begin{bmatrix}
            sx  &   0   &   0\\
            0   &   sy  &   0\\
            0   &   0   &   1
        \end{bmatrix}
        *
        \begin{bmatrix}
            cos\theta   &   sin\theta   &   0\\
            -sin\theta  &   cos\theta   &   0\\
            0           &   0           &   1
        \end{bmatrix}
        =
        \begin{bmatrix}
            sx*cos\theta    &   sx*sin\theta    &   0\\
            -sy*sin\theta   &   sy*cos\theta    &   0\\
            0               &   0               &   1
        \end{bmatrix}
        \end{equation}
    }
    \end {enumerate}
}

\end {document}
