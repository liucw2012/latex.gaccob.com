\begin {document}

\title {\ZHH \huge 学习ejoy2d——shader}
\author {\small gaccob}
\date {\small 2014 年 2 月 21 日}
\maketitle


\section {\ZHH shader简单介绍} {
    {wiki上这么描述: shader(着色器)指一组供计算机图形资源在执行渲染任务时使用的指令. shader是render的一部分, 运行在GPU上, 负责计算目标颜色. OpenGL从1.5开始继承一种类C的着色语言, 称为OpenGL Shader Language.}\par
    {shader分两种, 一种是顶点shader(OpenGL中是vertex shader), 目的是计算顶点位置, 为后期像素渲染做准备; 一种是像素shader(OpenGL中是fragment shader), 以像素为单位, 计算光照和颜色. }\par
}


\section {\ZHH ejoy2d.shader数据结构} {
    \begin{lstlisting}[language=C]
    // `\color{gray} 分别是screen 和 texture 的坐标`
    struct vertex {
        float vx;
        float vy;
        float tx;
        float ty;
        uint8_t rgba[4];
    };

    // `\color{gray} 1个quad, 4个顶点`
    struct quad {
        struct vertex p[4];
    };

    // `\color{gray} 这个东东处理了所有渲染部分的工作`
    struct render_state {

        // `\color{gray} 当前的shader program`
        int current_program;

        // `\color{gray} ejoy2d支持最多6种 shader program, 这个会在lua中定义`
        struct program program[MAX_PROGRAM];

        // texture id (OpenGL id)
        int tex;

        // `\color{gray} 需要渲染的quad的数量, 在rs\_commit()计算时需要用到`
        int object;

        // `\color{gray} 默认的blend方式 (这个下面代码有描述), 该值为0; 自定义blend方式时, 这个值=1`
        int blendchange;

        // `\color{gray} 顶点buffer`
        GLuint vertex_buffer;

        // `\color{gray} 索引buffer`
        GLuint index_buffer;

        // `\color{gray} 最多64个quad`
        struct quad vb[MAX_COMMBINE];
    };

    // `\color{gray} 全局渲染状态机`
    static struct render_state *RS = NULL;
    \end{lstlisting}
}


\section {\ZHH ejoy2d.shader初始化} {

    \begin{lstlisting}[language=C]

    // `\color{gray} 初始化shader, 这个会在程序启动时调用`
    void
    shader_init() {
        assert(RS == NULL);
        struct render_state * rs = (struct render_state *) malloc(sizeof(*rs));
        memset(rs, 0 , sizeof(*rs));
        rs->current_program = -1;

        // `\color{gray} 设置颜色混合的模式`
        // `\color{gray} ejoy2d还提供了shader\_defaultblend()和shader\_blend()接口来操作blend方式`
        rs->blendchange = 0;
        glBlendFunc(GL_ONE, GL_ONE_MINUS_SRC_ALPHA);

        // `\color{gray} 索引buffer`
        glGenBuffers(1, &rs->index_buffer);
        glBindBuffer(GL_ELEMENT_ARRAY_BUFFER, rs->index_buffer);

        GLubyte idxs[6 * MAX_COMMBINE];
        int i;
        for (i=0;i<MAX_COMMBINE;i++) {
            idxs[i*6] = i*4;
            idxs[i*6+1] = i*4+1;
            idxs[i*6+2] = i*4+2;
            idxs[i*6+3] = i*4;
            idxs[i*6+4] = i*4+2;
            idxs[i*6+5] = i*4+3;
        }

        // `\color{gray} GL\_STATIC\_DRAW表示索引是固定的`
        // `\color{gray} 上面的索引idxs, 实际上是将quad的4个顶点, 转为两个三角面, 节约了2个冗余顶点`
        glBufferData(GL_ELEMENT_ARRAY_BUFFER, 6*MAX_COMMBINE, idxs, GL_STATIC_DRAW);

        // `\color{gray} 顶点buffer, 这里的buffer会在程序运行中实时加载进来`
        glGenBuffers(1, &rs->vertex_buffer);
        glBindBuffer(GL_ARRAY_BUFFER, rs->vertex_buffer);

        glEnable(GL_BLEND);

        RS = rs;
    }

    \end{lstlisting} \par

    {这里值得说一下的是OpenGL的颜色混合方式, 假设源颜色(Rs, Gs, Bs, As), 目标颜色为(Rd, Gd, Bd, Ad), OpenGL分别讲源颜色和目标颜色乘一个系数, 就得到了混合的结果. 这里的系数就由glBlendFunc()指定. }\par

    {第一个参数GL\_ONE表示使用1.0作为源颜色的系数, 第二个参数, GL\_ONE\_MINUS\_SRC\_ALPHA表示以1.0减去As的值作为目标颜色的系数. } \par

    {具体的细节可以参考这一篇文章\href{http://blog.csdn.net/aurora_mylove/article/details/1700540}{《颜色混合opengl》}, 这里不再赘述. }\par

}

\section {\ZHH ejoy2d.shader的加载} {

    {在前文讲过, render\_state维护了一个预先加载的shader program数组. 可以从shader.lua中读到, shader程序有: sprite\_fs, sprite\_vs, text\_fs, text\_edge\_fs, gray\_fs 以及 color\_fs, fs和vs组合后有5种shader. }\par

    \begin{lstlisting}[language=lua]
    -- `\color{gray} lua 中的shader name`
    local shader_name = {
        NORMAL = 0,
        TEXT = 1,
        EDGE = 2,
        GRAY = 3,
        COLOR = 4,
    }

    -- `\color{gray} 在init时加载全部5种shader`
    function shader.init()
        s.load(shader_name.NORMAL, PRECISION .. sprite_fs, PRECISION .. sprite_vs)
        s.load(shader_name.TEXT, PRECISION .. text_fs, PRECISION .. sprite_vs)
        s.load(shader_name.EDGE, PRECISION .. text_edge_fs, PRECISION .. sprite_vs)
        s.load(shader_name.GRAY, PRECISION .. gray_fs, PRECISION .. sprite_vs)
        s.load(shader_name.COLOR, PRECISION .. color_fs, PRECISION .. sprite_vs)
    end
    \end{lstlisting}

    \begin{lstlisting}[language=C]

    // `\color{gray} 编译shader代码`
    static GLuint
    compile(const char * source, int type) {
        ......
    }

    // `\color{gray} 链接编译后的shader`
    static void
    link(struct program *p) {
        ......
    }

    // `\color{gray} 如果shader中存在addi, 设置为对应的color值`
    static void
    set_color(GLint addi, uint32_t color) {
        if (addi == -1)
            return;
        if (color == 0) {
            glUniform3f(addi, 0,0,0);
        } else {
            float c[3];
            c[0] = (float)((color >> 16) & 0xff) / 255.0f;
            c[1] = (float)((color >> 8) & 0xff) / 255.0f;
            c[2] = (float)(color & 0xff ) / 255.0f;
            glUniform3f(addi, c[0],c[1],c[2]);
        }
    }

    // `\color{gray} 加载shader program`
    static void
    program_init(struct program * p, const char *FS, const char *VS) {
        // Create shader program.
        p->prog = glCreateProgram();

        // `\color{gray} 编译FS, 像素shader`
        GLuint fs = compile(FS, GL_FRAGMENT_SHADER);
        if (fs == 0) {
            fault("Can't compile fragment shader");
        } else {
            glAttachShader(p->prog, fs);
        }

        // `\color{gray} 编译VS, 顶点shader`
        GLuint vs = compile(VS, GL_VERTEX_SHADER);
        if (vs == 0) {
            fault("Can't compile vertex shader");
        } else {
            glAttachShader(p->prog, vs);
        }

        // `\color{gray} 绑定顶点shader中的attribute 到这里的ATRRIB\_*变量`
        // `\color{gray} 这里的position, texcoord和color 是sprites\_vs shader中的attribute`
        glBindAttribLocation(p->prog, ATTRIB_VERTEX, "position");
        glBindAttribLocation(p->prog, ATTRIB_TEXTCOORD, "texcoord");
        glBindAttribLocation(p->prog, ATTRIB_COLOR, "color");

        // `\color{gray} 链接`
        link(p);

        // `\color{gray} 获取像素shader中的uniform变量 additive, (一个偏移量, 默认是0)`
        p->additive = glGetUniformLocation(p->prog, "additive");
        p->arg = 0;
        set_color(p->additive, 0);

        // `\color{gray} 删除shader, link完之后, fs和vs就不用了.`
        // `\color{gray} 跟平时c的编译其实很类似, link成lib或者bin之后, .o文件就不用了.`
        glDetachShader(p->prog, fs);
        glDeleteShader(fs);
        glDetachShader(p->prog, vs);
        glDeleteShader(vs);
    }

    // `\color{gray} 加载shader程序`
    void
    shader_load(int prog, const char *fs, const char *vs) {
        struct render_state *rs = RS;
        assert(prog >=0 && prog < MAX_PROGRAM);
        struct program * p = &rs->program[prog];
        assert(p->prog == 0);
        program_init(p, fs, vs);
    }

    // `\color{gray} 卸载shader`
    // `\color{gray} 其实我认为不仅是unload 还有release 为什么不写成两个函数呢? `
    void
    shader_unload() {
        if (RS == NULL) {
            return;
        }
        int i;

        // `\color{gray} 卸载所有的shader程序`
        for (i=0;i<MAX_PROGRAM;i++) {
            struct program * p = &RS->program[i];
            if (p->prog) {
                glDeleteProgram(p->prog);
            }
        }

        // `\color{gray} 删除顶点buffer和索引buffer`
        glDeleteBuffers(1,&RS->vertex_buffer);
        glDeleteBuffers(1,&RS->index_buffer);

        // `\color{gray} 释放全局渲染状态机`
        free(RS);
        RS = NULL;
    }

    \end{lstlisting}
}

\section {\ZHH ejoy2d.shader的渲染} {

    {ejoy2d中用了OpenGL的VAO来做渲染, 具体的细节可以参考这一片文章 \href{http://www.cppblog.com/init/archive/2012/02/21/166098.html}{《OpenGL.Vertex Array Object (VAO)》}. }\par

    \begin{lstlisting}[language=C]
    // `\color{gray} 渲染的过程, 这里对quad利用索引buffer做了一些优化(节省了冗余的vertex)`
    static void
    rs_commit() {
        if (RS->object == 0)
            return;

        // `\color{gray} 顶点buffer, GL\_DYNAMIC\_DRAW说明这里每一帧可能会渲染多次`
        glBindBuffer(GL_ARRAY_BUFFER, RS->vertex_buffer);
        glBufferData(GL_ARRAY_BUFFER, sizeof(struct quad) * RS->object, RS->vb, GL_DYNAMIC_DRAW);

        // `\color{gray} 指定ATTRIB\_VERTEX -> shader中的position`
        // `\color{gray} 2个float, 间隔一个struct vertex, offset=0, 对应vertex->vx, vertex->vy`
        glEnableVertexAttribArray(ATTRIB_VERTEX);
        glVertexAttribPointer(ATTRIB_VERTEX, 2, GL_FLOAT, GL_FALSE, sizeof(struct vertex), BUFFER_OFFSET(0));

        // `\color{gray} 指定ATTRIB\_TEXTCOORD -> shader中的texcoord`
        // `\color{gray} 2个float, 间隔一个struct vertex, offset=8=2*sizeof(GL\_FLOAT) 对应vertex->tx, vertex->ty`
        glEnableVertexAttribArray(ATTRIB_TEXTCOORD);
        glVertexAttribPointer(ATTRIB_TEXTCOORD, 2, GL_FLOAT, GL_FALSE, sizeof(struct vertex), BUFFER_OFFSET(8));

        // `\color{gray} 指定ATTRIB\_COLOR -> shader中的color`
        // `\color{gray} 4个unsigned byte, 间隔一个struct vertex, offset=16=4*sizeof(GL\_FLOAT) 对应vertex->rgba`
        glEnableVertexAttribArray(ATTRIB_COLOR);
        glVertexAttribPointer(ATTRIB_COLOR, 4, GL_UNSIGNED_BYTE, GL_TRUE, sizeof(struct vertex), BUFFER_OFFSET(16));

        // `\color{gray} 使用顶点buffer绘制, count=6*object, 与索引buffer一致`
        glDrawElements(GL_TRIANGLES, 6 * RS->object, GL_UNSIGNED_BYTE, 0);
        RS->object = 0;
    }

    // `\color{gray} 渲染一个quad`
    void
    shader_draw(const float vb[16], uint32_t color) {
        struct quad *q = RS->vb + RS->object;
        int i;
        for (i=0;i<4;i++) {
            q->p[i].vx = vb[i*4+0];
            q->p[i].vy = vb[i*4+1];
            q->p[i].tx = vb[i*4+2];
            q->p[i].ty = vb[i*4+3];
            q->p[i].rgba[0] = (color >> 16) & 0xff;
            q->p[i].rgba[1] = (color >> 8) & 0xff;
            q->p[i].rgba[2] = (color) & 0xff;
            q->p[i].rgba[3] = (color >> 24) & 0xff;
        }
        if (++RS->object >= MAX_COMMBINE) {
            rs_commit();
        }
    }


    // `\color{gray} 渲染一个polygon, 这里调用glDrawArrays(GL\_TRIANGLE\_FAN, ...)来实现绘制`
    // `\color{gray} GL\_TRIANGLE\_FAN与GL\_TRIANGLE\_STRIP的区别, 可以参考\href{http://blog.csdn.net/xiajun07061225/article/details/7455283}{这篇文章}`
    void
    shader_drawpolygon(int n, const float *vb, uint32_t color) {
        rs_commit();
        struct vertex p[n];
        int i;
        for (i=0;i<n;i++) {
            p[i].vx = vb[i*4+0];
            p[i].vy = vb[i*4+1];
            p[i].tx = vb[i*4+2];
            p[i].ty = vb[i*4+3];
            p[i].rgba[0] = (color >> 16) & 0xff;
            p[i].rgba[1] = (color >> 8) & 0xff;
            p[i].rgba[2] = (color) & 0xff;
            p[i].rgba[3] = (color >> 24) & 0xff;
        }

        glBindBuffer(GL_ARRAY_BUFFER, RS->vertex_buffer);
        glBufferData(GL_ARRAY_BUFFER, sizeof(struct vertex) * n, (void*)p, GL_DYNAMIC_DRAW);

        glEnableVertexAttribArray(ATTRIB_VERTEX);
        glVertexAttribPointer(ATTRIB_VERTEX, 2, GL_FLOAT, GL_FALSE, sizeof(struct vertex), BUFFER_OFFSET(0));
        glEnableVertexAttribArray(ATTRIB_TEXTCOORD);
        glVertexAttribPointer(ATTRIB_TEXTCOORD, 2, GL_FLOAT, GL_FALSE, sizeof(struct vertex), BUFFER_OFFSET(8));
        glEnableVertexAttribArray(ATTRIB_COLOR);
        glVertexAttribPointer(ATTRIB_COLOR, 4, GL_UNSIGNED_BYTE, GL_TRUE, sizeof(struct vertex), BUFFER_OFFSET(16));
        glDrawArrays(GL_TRIANGLE_FAN, 0, n);
    }

    \end{lstlisting}
}

\section {\ZHH ejoy2d.shader的lua接口} {

    {shader的lua接口代码, 都在lib/lshader.c中. }\par

    \begin{lstlisting}[language=C]
    int
    ejoy2d_shader(lua_State *L) {
        luaL_Reg l[] = {
            {"load", lload},
            {"unload", lunload},
            {"draw", ldraw},
            {"blend", lblend},
            {"version", lversion},
            {NULL,NULL},
        };
        luaL_newlib(L,l);
        return 1;
    }
    \end{lstlisting}

    {这里基本就是从lua读参数, 然后调用C接口实现, 源码很清晰, 就不列举了. 单独说一下shader.draw这个接口.}\par

    \begin{lstlisting}[language=C]
    // `\color{gray} 栈里的lua参数:`
    // `\color{gray}     int texture`
    // `\color{gray}     table float[16]: 先texture(tx,ty)列表 再screen(vx, vy)列表`
    // `\color{gray}     uint32\_t color`
    // `\color{gray}     uint32\_t additive`
    static int
    ldraw(lua_State *L) {
        int tex = (int)luaL_checkinteger(L,1);
        int texid = texture_glid(tex);
        if (texid == 0) {
            lua_pushboolean(L,0);
            return 1;
        }
        luaL_checktype(L, 2, LUA_TTABLE);
        uint32_t color = 0xffffffff;

        if (!lua_isnoneornil(L,3)) {
            color = (uint32_t)lua_tounsigned(L,3);
        }

        // `\color{gray} 设置program和additive`
        uint32_t additive = (uint32_t)luaL_optunsigned(L,4,0);
        shader_program(PROGRAM_PICTURE,additive);

        // `\color{gray} 设置texture`
        shader_texture(texid);

        // `\color{gray} 每个vertex有vx,vy,tx,ty, 所以必然是4的倍数`
        int n = lua_rawlen(L, 2);
        int point = n/4;
        if (point * 4 != n) {
            return luaL_error(L, "Invalid polygon");
        }

        float vb[n];
        int i;
        for (i=0;i<point;i++) {
            // `\color{gray} get (tx, ty) (vx, vy)`
            lua_rawgeti(L, 2, i*2+1);
            lua_rawgeti(L, 2, i*2+2);
            lua_rawgeti(L, 2, point*2+i*2+1);
            lua_rawgeti(L, 2, point*2+i*2+2);
            float tx = lua_tonumber(L, -4);
            float ty = lua_tonumber(L, -3);
            float vx = lua_tonumber(L, -2);
            float vy = lua_tonumber(L, -1);
            lua_pop(L,4);

            // `\color{gray} screen 坐标系 normalize`
            screen_trans(&vx,&vy);

            // `\color{gray} texture 坐标系 normalize`
            texture_coord(tex, &tx, &ty);

            // `\color{gray} 这个是为了坐标系对齐, screen坐标系原点(0, 0)是屏幕中间, 现在调整到左下角`
            vb[i*4+0] = vx + 1.0f;
            vb[i*4+1] = vy - 1.0f;
            vb[i*4+2] = tx;
            vb[i*4+3] = ty;
        }

        // quad
        if (point == 4) {
            shader_draw(vb, color);
        }
        // polygon
        else {
            shader_drawpolygon(point, vb, color);
        }
        return 0;
    }
    \end{lstlisting}
}

\section {\ZHH ejoy2d.shader的sample} {
    {可以参考examples/ex02.lua. }\par
    \begin{lstlisting}[language=lua]

    local shader = require "ejoy2d.shader"

    -- ` \color{gray} 这里是直接调用shader绘制的sample, 注意一下这里屏幕坐标的中心原点`
    function game.drawframe()
	-- use shader.draw to draw a polygon to screen (for debug use)
	shader.draw(TEXID, {
		88, 0, 88, 45, 147, 45, 147, 0,	-- texture coord
		-958, -580, -958, 860, 918, 860, 918, -580, -- screen coord, 16x pixel, (0,0) is the center of screen
	})
end

    \end{lstlisting}
}

\end {document}
