\begin{document}

\title {\ZHH \huge 编码标准}
\author {\small gaccob}
\date {\small 2012 年 9 月 7 日}
\maketitle

\section*{\ZHH ASCII}{
    {最早的编码标准, 1个字节, 128个字符集, 0×00-0x7F. }\par
}

\section*{\ZHH EASCII}{
    {Extended ASCII, 延伸美国标准信息交换码, EASCII码比ASCII码扩充出来的符号包括表格符号, 计算符号, 希腊字母和特殊的拉丁符号. 相比于ASCII码, 增加了128–255的这一段. }\par
}

\section*{\ZHH GB2312}{
    {GB2312开始支持了汉字, 1981年5月开始在国内实行. }\par
    {GB2312基于分区处理, 编码时采用2字节, 兼容ASCII码,  高字节(0xA1-0xF7, =区号+0xA0),低字节(0xA1-0xFE), 高位都是1, 所以能兼容ASCII码. }\par
}

\section*{\ZHH BIG5}{
    {BIG5开始支持了繁体汉字. 与GB2312一样, 用两个字节来为每个字符编码, 第一个字节称为"高位字节", 第二个字节称为"低位字节". }\par
}

\section*{\ZHH Unicode}{
    {为了解决传统的字符编码方案的局限而产生的, 由国际标准化组织(ISO)发布的编码标准. }\par
    {Unicode是一种编码标准, 它的实现有很多种: UTF(Unicode Translation Format)的UTF8和UTF16, GB系列的GBK和GB18030等. }\par
    \begin{itemize}
    \item{\ZHH \small GBK}\par{
        {在unicode推出之后, 中国大陆制定了GB13000标准, 几乎等同于unicode1.1.}\par
        {1995年微软利用了GB2312中未使用的编码空间, 收录了GB13000中的所有字符制定了汉字内码扩展规范GBK. }\par
        {GBK编码最多使用2个字节:字节为00-7F的表示与ASCII完全一致;最高位为1的字节表示2个字节编码, 高字节范围为81-FE, 低字节范围为40-FE, 或者80-FE.}\par
    }
    \item{\ZHH \small GB18030}\par{
        {在2000年, 电子工业标准化研究所起草了GB18030标准, 项目代号"GB 18030-2000", 全称《信息技术-信息交换用汉字编码字符集-基本集的扩充》.}\par
        {GB18030收录了GBK中的所有字符, 并将Unicode中其他中文字符(少数民族文字, 偏僻字)也一并收录进来重新编码. }\par
        {采用多字节编码, 每个字符由1或2或4个字节进行编码. }\par
    }
    \item{\ZHH \small UTF8}\par{
        {UTF8文件头"EF BB BF"(BOM-byte order mark), 但不是所有的UTF8文件都有. }\par
        {UTF-8 是以8位为单元对原始Unicode码进行编码, 并规定:多字节码以转换后第1个字节起头的连续"1"的数目, 表示转换成几个字节:"110"连续两个"1", 表示转换结果为2个字节, "1110"表示3个字节, 跟随在标记位之后的"0", 其作用是分隔标记位和字符码位. }\par
        {第2~第4个字节的起头两个位固定设置为"10", 也作为标记, 剩下的6个位才做为字符码位使用. 2字节UTF-8码剩下11个字符码位, 可用以转换0080~07FF的原始字符码;3字节剩下16个字符码位, 可用以转换0800~FFFF的原始字符码, 由此类推}\par
        \begin{lstlisting}[language=bash]
     `原始码(16进制) UTF-8编码(二进制)`
     0000 - 007F 0xxxxxxx
     0080 - 07FF 110xxxxx 10xxxxxx
     0800 - FFFF 1110xxxx 10xxxxxx  10xxxxxx
        \end{lstlisting}
        {ASCII码<007F, 编为1个字节的UTF8码. }\par
        {汉字的 Unicode编码范围为0800-FFFF, 所以被编为3个字节的UTF8码. }\par

        {下面这一段是根据utf8字符获得unicode的解析代码:}\par
        \begin{lstlisting}[language=C]
    int _get_unicode(const char* str, int n)
    {
        int i;
        int unicode = str[0] & ((1 << (8-n)) - 1);
        for (i=1; i<n; i++) {
            unicode = unicode << 6 | ((uint8_t)str[i] & 0x3f);
        }
        return unicode;
    }

    int get_unicode(char** utf8, int* unicode)
    {
        uint8_t c;
        if (!utf8 || !(*utf8) || !unicode) {
            return -1;
        }
        if (**utf8 == 0) {
            return -1;
        }
        c = (*utf8)[0];
        if ((c & 0x80) == 0) {
            *unicode = _get_unicode(*utf8, 1);
            *utf8 += 1;
        } else if ((c & 0xe0) == 0xc0) {
            *unicode = _get_unicode(*utf8, 2);
            *utf8 += 2;
        } else if ((c & 0xf0) == 0xe0) {
            *unicode = _get_unicode(*utf8, 3);
            *utf8 += 3;
        } else if ((c & 0xf8) == 0xf0) {
            *unicode = _get_unicode(*utf8, 4);
            *utf8 += 4;
        } else if ((c & 0xfc) == 0xf8) {
            *unicode = _get_unicode(*utf8, 5);
            *utf8 += 5;
        } else if ((c & 0xfe) == 0xfc) {
            *unicode = _get_unicode(*utf8, 6);
            *utf8 += 6;
        } else {
            return -1;
        }
        return 0;
    }
        \end{lstlisting}
    }
    \end{itemize}
}

{\ZHH 参考文章}\par
{1. \href{http://djt.qq.com/article/view/658?ADTAG=email.InnerAD.weekly.20130902}{字符编码的前世今生}}\par
{2. \href{http://baike.baidu.com/view/42488.htm}{http://baike.baidu.com/view/42488.htm}}\par
{3. \href{http://zh.wikipedia.org/wiki/Unicode}{http://zh.wikipedia.org/wiki/Unicode}}\par

\end{document}
